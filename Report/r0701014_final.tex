\documentclass[a4paper,10pt]{article}
\usepackage[utf8]{inputenc}

\usepackage[english]{babel}
\usepackage[dvinames]{xcolor}
\usepackage[compact,small]{titlesec}
\usepackage{booktabs}
\usepackage{multirow}
\usepackage{amsfonts,amsmath,amssymb}
\usepackage{marginnote}
\usepackage[top=1.8cm, bottom=1.8cm, outer=1.8cm, inner=1.8cm, heightrounded, marginparwidth=2.5cm, marginparsep=0.5cm]{geometry}
\usepackage{enumitem}
\setlist{noitemsep,parsep=2pt}
\newcommand{\highlight}[1]{\textcolor{kuleuven}{#1}}
\usepackage{pythonhighlight}
\usepackage{cleveref}
\usepackage{graphicx}
\usepackage[nottoc]{tocbibind}
\usepackage{url}

\newcommand{\nextyear}{\advance\year by 1 \the\year\advance\year by -1}
\newcommand{\thisyear}{\the\year}
\newcommand{\deadlineGroup}{November 27, \thisyear{} at 16:00 CET}
\newcommand{\deadlineCode}{December 18, \thisyear{} at 16:00 CET}
\newcommand{\deadlineReport}{January 4, \nextyear{} at 16:00 CET}

\newcommand{\ReplaceMe}[1]{{\color{blue}#1}}
\newcommand{\RemoveMe}[1]{{\color{purple}#1}}

\setlength{\parskip}{5pt}

%opening
\title{Evolutionary Algorithms: Final report}
\author{Sander Prenen (r0701014)}

\begin{document}
\fontfamily{ppl}
\selectfont{}

\maketitle

\section{\RemoveMe{Formal requirements}} \label{sec_this}

\RemoveMe{The report is structured for fair and efficient grading of over 100 individual projects in the space of only a few days. Please respect the exact structure of this document. You are allowed to remove sections \ref{sec_this} and \ref{sec_other}. Brevity is the soul of wit: a good report will be \textbf{around $7$ pages} long. The hard limit is 10 pages. 

It is recommended that you use this \LaTeX{} template, but you are allowed to reproduce it with the same structure in a WYSIWYG-editor. The purple text containing our evaluation criteria can be removed. You should replace the blue text with your discussion. \textbf{The questions we ask in blue are there to guide which topics to discuss}, rather than an exact list of questions that must be answered. Feel free to add more items to discuss.

This report should be uploaded to Toledo by \deadlineReport. It must be in the \textbf{Portable Document Format} (pdf) and must be named \texttt{r0123456\_intermediate.pdf}, where r0123456 should be replaced with your student number.}

\section{Metadata}

\begin{itemize}
 \item \textbf{Group members during group phase:} Karel Everaert and Matthias Maeyens
 \item \textbf{Time spent on group phase:} 15 hours
 \item \textbf{Time spent on final code:} \ReplaceMe{40 hours}
 \item \textbf{Time spent on final report:} \ReplaceMe{10 hours}
\end{itemize}

\section{Modifications since the group phase}

\RemoveMe{\textbf{Goal:} Based on this section, we will evaluate insofar as you are able to analyse common problems arising in the design and implementation of evolutionary algorithms and your ability to effectively solve them.}

\subsection{Main improvements}

\ReplaceMe{List the main changes that you implemented since the group phase. You do not need to explain the employed techniques in detail; for this, you should refer to the appropriate subsection of section 4 of the report.}

\paragraph{Initialization of the population:}A greedy heuristic was used to initialize the population in order to start from a fitter population than the random population used in the group phase.

\paragraph{Schemes:} Different schemes were carried out for all the main operators in the evolutionary algorithm (selection, mutation, recombination, elimination).

\paragraph{Implementation of a local search operator:} A local search operator is implemented to improve the candidate solutions and speed up the convergence.

\paragraph{Parallelization of the most expensive parts of the algorithm:} Computationally expensive parts of the algorithm like the local search operator have been parallelized by using the \pyth{multiprocessing} library.


\subsection{Issues resolved}
\ReplaceMe{Recall the list of issues from the group phase. Describe how you solved these issues in the individual phase.}

\paragraph{Mutation:} The mutation operator used in the group phase had little on larger problem sizes. A number of different mutation operators were implemented to cope with this problem.

\paragraph{Elimination scheme:} The algorithm created in the group phase had the problem that the fittest candidate solution got mutated unwillingly, making divergence from the optimal tour possible. This issue was fixed by integrating an elitism elimination scheme to always keep the fittest individual in the next generation.

\paragraph{Diversity promotion:} The implementation in the group phase suffered from a lack of diversity. This caused the algorithm to converge to local optima. This problem was solved by adding a diversity promotion scheme to the evolutionary algorithm.

\paragraph{Recombination operator:} The recombination operator used in the group phase did not preserve the order of the nodes in the tour, but rather tried to preserve the position of the nodes. Several new recombination operators were created to deal with this problem.

\section{Final design of the evolutionary algorithm} 

\RemoveMe{\textbf{Goal:} Based on this section, we will evaluate insofar as you are able to design and implement an advanced, effective evolutionary algorithm for solving a model problem.}

\ReplaceMe{In this section, you should describe all components of your final evolutionary algorithm and how they fit together.}

\subsection{Representation}

\ReplaceMe{How do you represent the candidate solutions? What is your motivation to choose this one? What other options did you consider? How did you implement this specifically in Python (e.g., a list, set, numpy array, etc)?}

An individual is represented as a 1D \pyth{numpy} array with the order of the visited nodes. A \pyth{numpy} array is used rather than a plain python list, because of the more efficient manner in which these arrays get stored in memory and the vectorized operations that are available in the \pyth{numpy} library. A population is represented as a two dimensional array (each row is an individual). This choice was made for the same reasons as mentions above.

\subsection{Initialization}

\ReplaceMe{How do you initialize the population? How did you determine the number of individuals? Did you implement advanced initialization mechanisms (local search operators, heuristic solutions)? If so, describe them. Do you believe your approach maintains sufficient diversity? How do you ensure that your population enrichment scheme does not immediately take over the population? Did you implement other initialization schemes that did not make it to the final version? Why did you discard them? How did you determine the population size?}

The population is initialized by using a greedy algorithm. The algorithm starts in a node and picks the nearest not visited node as the next node. The starting points are decided at random. This is a variation on all nearest neighbors (ANN) as proposed by Kaabi and Harrath \cite{ann}. This heuristic method creates less diversity than random initialization. Therefore a diversity promotion scheme is implemented and described in section \ref{sec:diversity}. The random initialization method is still available by setting the boolean value of \texttt{use\_random\_initialization} to true. This version explores more of the search space, but it will never reach the heuristic solution for the bigger problems in the given time span.

\subsection{Selection operators}

\ReplaceMe{Which selection operators did you implement? If they are not from the slides, describe them. Can you motivate why you chose this one? Are there parameters that need to be chosen? Did you use an advanced scheme to vary these parameters throughout the iterations? Did you try other selection operators not included in the final version? Why did you discard them?}

The following selection operators were implemented: k-tournament selection, linear rank based selection and roulette wheel selection with geometric decaying selection pressure. Some of these selection operators depend on parameters chosen by the user. For k-tournament this parameter is $k$, the number of individuals to be randomly selected for the tournament. The roulette wheel selection depends on the starting selection pressure and the decay for this selection pressure. The linear rank based selection does not need any parameters to be chosen.\\
The final version of the evolutionary algorithm uses the roulette wheel selection with geometric decay. By doing a hyperparameter search, as described in section \ref{sec:hyperparameter}, the values of the starting selection pressure and the decay are set to \ReplaceMe{x and y} respectively. The generation of individuals in this operator is not done using inverse transform sampling, as this would require $\mathcal{O}(n)$ time per sample. Instead the Vose's alias method is used as explained by Schwarz \cite{schwarz, vose}. This method is able to sample in constant time, but uses $\mathcal{O}(n)$ time to initialize. This initialization can be reused to select all the individuals during the current generation.\\
A fitness proportionate selection scheme was briefly introduced but discarded due to the problems mentioned in the slides.

\subsection{Mutation operators}

\ReplaceMe{Which mutation operators did you implement? If they are not from the slides, describe them. How do you choose among several mutation operators? Do you believe it will introduce sufficient randomness? Can that be controlled with parameters? Do you use self-adaptivity? Do you use any other advanced parameter control mechanisms (e.g., variable across iterations)? Did you try other mutation operators not included in the final version? Why did you discard them?}

The mutation operator that existed in the group phase, the sequence swap operator, was simplified to a simple swap operator that swaps two nodes at random. This operator had little effect on larger problem instances due to its proportionally small range. Therefore three different operators were introduced: reverse sequence mutation (RSM), partial shuffle mutation (PSM) and a hybrid of the two called hybridizing PSM RSM mutation (HPRM) \cite{hprm}.\\
RSM chooses two nodes in the individual at random and reverses the sequence between the two nodes. PSM is similar but randomly shuffles the sequence between the nodes, thus introducing more randomness. HPRM is a combination of the two where the sequence between the nodes is reversed element-wise and shuffled after each reversion.\\
All the operators discussed cannot be controlled by a parameter to tune the amount of randomness they introduce. The final evolutioanry algorithm uses \ReplaceMe{x} because it proved the best in the numerical experiments. The operator is applied with a probability $\alpha$ to an individual, with $\alpha = 0.2 *\frac{ \textit{mean objective}}{\textit{mean objective } + \textit{ best objective}}$. This idea is based on the idea from Kaabi and Harrath \cite{ann}. 

\subsection{Recombination operators}

\ReplaceMe{Which recombination operators did you implement? If they are not from the slides, describe them. How do you choose among several recombination operators? Why did you choose these ones specifically? Explain how you believe that these operators can produce offspring that combine the best features from their parents. How does your operator behave if there is little overlap between the parents? Can your recombination be controlled with parameters; what behavior do they change? Do you use self-adaptivity? Do you use any other advanced parameter control mechanisms (e.g., variable across iterations)? Did you try other recombination operators not included in the final version? Why did you discard them? Did you consider recombination with arity strictly greater than 2?}



\subsection{Elimination operators}

\ReplaceMe{Which elimination operators did you implement? If they are not from the slides, describe them. Why did you select this one? Are there parameters that need to be chosen? Did you use an advanced scheme to vary these parameters throughout the iterations? Did you try other elimination operators not included in the final version? Why did you discard them?} 

\subsection{Local search operators}

\ReplaceMe{What local search operators did you implement? Describe them. Did they cause a significant improvement in the performance of your algorithm? Why (not)? Did you consider other local search operators that did not make the cut? Why did you discard them? Are there parameters that need to be determined in your operator? Do you use an advanced scheme to determine them (e.g., adaptive or self-adaptive)?}

\subsection{Diversity promotion mechanisms} \label{sec:diversity}

\ReplaceMe{Did you implement a diversity promotion scheme? If yes, which one? If no, why not? Describe the mechanism you implemented. In what sense does the mechanism improve the performance of your evolutionary algorithm? Are there parameters that need to be determined? Did you use an advanced scheme to determine them?}

\subsection{Stopping criterion}

\ReplaceMe{Which stopping criterion did you implement? Did you combine several criteria?}

During the development phase, a combination of two criteria was used. The first one was a limit on the number of generations (500 in this case) and the second one was the number of generations in which the best solutions had not changed. The former criterion was obsolete after for the larger problems due to the computationally expensive local search operator. Since the elitism scheme was used in the final version of the algorithm, the stopping criterion has been dropped in order to fully use the given time span.

\subsection{The main loop}

\ReplaceMe{Describe the main loop of your evolutionary algorithm using a clear picture (preferred) or high-level pseudocode. In what order do you apply the various operators? Why that order? If you are using several selection, mutation, recombination, elimination, and local search operators, describe how you choose among the possibilities. Are you selecting/eliminating all individuals in parallel, or one by one? With or without replacement?}

\subsection{Parameter selection} \label{sec:hyperparameter}

\ReplaceMe{For all of the parameters that are not automatically determined by adaptivity or self-adaptivity (as you have described above), describe how you determined them. Did you perform a hyperparameter search? How did you do this? How did you determine these parameters would be valid both for small and large problem instances?}

\subsection{Other considerations}

\ReplaceMe{Did you consider other items not listed above, such as elitism, multiobjective optimization strategies (e.g., island model, pareto front approximation), a parallel implementation, or other interesting computational optimizations (e.g. using advanced algorithms or data structures)? You can describe them here or add additional subsections as needed.}


\section{Numerical experiments}

\RemoveMe{\textbf{Goal:} Based on this section and our execution of your code, we will evaluate the performance (time, quality of solutions) of your implementation and your ability to interpret and explain the results on benchmark problems.}

\subsection{Metadata}

\ReplaceMe{What parameters are there to choose in your evolutionary algorithm? Which fixed parameter values did you use for all experiments below? If some parameters are determined based on information from the problem instance (e.g., number of cities), also report their specific values for the problems below.

Report the main characteristics of the computer system on which you ran your evolutionary algorithm. Include the processor or CPU (including the number of cores and clock speed), the amount of main memory, and the version of Python 3.}


\subsection{tour29.csv}

\ReplaceMe{Run your algorithm on this benchmark problem (with the 5 minute time limit from the Reporter). Include a typical convergence graph, by plotting the mean and best objective values in function of the time (for example based on the output of the Reporter class). 

What is the best tour length you found? What is the corresponding sequence of cities? 

Interpret your results. How do you rate the performance of your algorithm (time, memory, speed of convergence, diversity of population, quality of the best solution, etc)? Is your solution close to the optimal one?

Solve this problem 1000 times and record the results. Make a histogram of the final mean fitnessess and the final best fitnesses of the 1000 runs. Comment on this figure: is there a lot of variability in the results, what are the means and the standard deviations?}

\subsection{tour100.csv}

\ReplaceMe{Run your algorithm on this benchmark problem (with the 5 minute time limit from the Reporter). Include a typical convergence graph, by plotting the mean and best objective values in function of the time (for example based on the output of the Reporter class). 

What is the best tour length you found in each case? 

Interpret your results. How do you rate the performance of your algorithm (time, memory, speed of convergence, diversity of population, quality of the best solution, etc)? Is your solution close to the optimal one?}

\subsection{tour194.csv}

\ReplaceMe{Run your algorithm on this benchmark problem (with the 5 minute time limit from the Reporter). Include a typical convergence graph, by plotting the mean and best objective values in function of the time (for example based on the output of the Reporter class). 

What is the best tour length you found? 

Interpret your results. How do you rate the performance of your algorithm (time, memory, speed of convergence, diversity of population, quality of the best solution, etc)? Is your solution close to the optimal one?}

\subsection{tour929.csv}

\ReplaceMe{Run your algorithm on this benchmark problem (with the 5 minute time limit from the Reporter). Include a typical convergence graph, by plotting the mean and best objective values in function of the time (for example based on the output of the Reporter class). 

What is the best tour length you found? 

Interpret your results. How do you rate the performance of your algorithm (time, memory, speed of convergence, diversity of population, quality of the best solution, etc)? Is your solution close to the optimal one? 

Did your algorithm converge before the time limit? How many iterations did you perform?}



\section{Critical reflection}

\RemoveMe{\textbf{Goal:} Based on this section, we will evaluate your understanding and insight into the main strengths and weaknesses of your evolutionary algorithms.}

\ReplaceMe{Describe the main lessons learned from this project. What do you think are the main strong points of evolutionary algorithms in general? Did you apply these strengths in this project? What are the main weaknesses of evolutionary algorithms and of your implementation in particular? Do you think these can be avoided or mitigated? How? Do you believe evolutionary algorithms are appropriate for this problem? Why (not)? What surprised you and why? What did you learn from this project?}

\section{Other comments} \label{sec_other}

\ReplaceMe{In case you think there is something important to discuss that is not covered by the previous sections, you can do it here. }

\bibliographystyle{unsrt}
\bibliography{mybib}

\end{document}
